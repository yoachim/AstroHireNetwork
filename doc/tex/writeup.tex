%\documentclass[12pt,preprint]{aastex}
\documentclass{emulateapj}
\usepackage{url}
%\usepackage{natbib}
%\usepackage{xspace}
\def\arcsec{$^{\prime\prime}$}
\bibliographystyle{apj}
\newcommand\degree{{^\circ}}
\newcommand\surfb{$\mathrm{mag}/\square$\arcsec}
\newcommand\Gyr{\rm{~Gyr}}
\newcommand\msun{\rm{M}_\odot}
\newcommand\kms{km s$^{-1}$}
\newcommand\al{$\alpha$}
\newcommand\ha{$\rm{H}\alpha$}
\newcommand\hb{$\rm{H}\beta$}



\shorttitle{Astronomy Publication Lifetimes}
\shortauthors{Yoachim et al.}

\begin{document}

\title{Publication Lifetimes of American Astronomy PhDs: A Post-2008 Collapse}

\author{Peter Yoachim\altaffilmark{1}, author 2, author 3 
  }
\altaffiltext{1}{Department of Astronomy, University of Washington, Box 351580,
Seattle WA, 98195; {yoachim@uw.edu} }


\begin{abstract}
blah blah blah, abstract

\end{abstract}
\keywords{employment}


\section{TO DO}
\begin{itemize}
\item{Just write text! }
\end{itemize}

\section{Introduction}

It is difficult to determine the state of the Astronomy job market.  Analysis of the science job market often lumps all sciences together, in which case Biologists dominate the statistics, or, if the sciences are separated, Astronomy is usually grouped with Physics (e.g., the Bureau of Labor and Statistics lists 20,000 Physicists and Astronomers jobs in 2014 \footnote{\url{http://www.bls.gov/ooh/life-physical-and-social-science/physicists-and-astronomers.htm}}), where the number of Physicists outnumbers the Astronomers.  The American Astronomical Society currently lists it's membership at around 7,000.

\citet{Seth09} build on the work of \citet{Metcalfe08} and looks at the rise of the postdoc era in Astronomy and Astrophysics.  They show that as the inflation-adjusted US-Astronomy budget doubled over 20 years, there was a modest increase in PhD production and a huge rise in the number of temporary postdoc positions.

In this paper, we use the SAO/NASA Astrophysics Data System (ADS) to explore the ramifications of that growing postdoc bubble.  While the NSF regularly surveys PhD recipients (e.g., \url{http://nsf.gov/statistics/nsf07305/}), even with high participation rates, surveys will tend to be biased against those that leave the field since they will be harder to contact.  Thus, we propose to use the publication record of PhDs to track the overall health of the field.  

Unfortunately, some faculty still espouse the theory that, ``... the [astronomy faculty job] market is no worse or better than it is has been for at least a decade or two'' (via Charfman \url{http://jjcharfman.tumblr.com/}).  The goal of this paper is to check this statement and quantify the astronomy job market by tracking the digital footprints Astronomers leave in ADS. 

XXX-description of ADS, how it includes peer-reviewed papers as well as things like conference proceedings, posters, grants, etc.

Example of current tracking here \url{https://www.aip.org/sites/default/files/statistics/employment/phd1yrlater-p-14.pdf}--note that they only get info on 48\% of physics PhD reciepients.  We expect that ADS is much more complete in listing PhD's than 48\%.


add citation to: http://www.ncbi.nlm.nih.gov/pmc/articles/PMC4309283/

Larson, Richard C., Navid Ghaffarzadegan, and Yi Xue. “Too Many PhD Graduates or Too Few Academic Job Openings: The Basic Reproductive Number R0 in Academia.” Systems research and behavioral science 31.6 (2014): 745–750. PMC. Web. 14 July 2016.

\section{Database Construction}\label{sec:db_construct}

All the code used for this paper is available on github\footnote{\url{https://github.com/yoachim/AstroHireNetwork}}

We use the excellent {\emph{Python Module to Interact with NASA's ADS that Doesn't Suck™}}\footnote{\url{https://github.com/andycasey/ads}} to access records from the SAO/NASA Astrophysics Data System XXX--figure out proper way to cite code. 

Our procedure for building a database of Astronomy and Astrophysics PhDs is as follows.  We select all PhD Thesis records for a given year and screen to include only those records that have an affiliation in the United States.

For each US PhD, we then query ADS for all entries with the PhD author's last name and first initial with a date limit of 7 years pre-PhD to the present (Jan 2016).

XXX--So, we looked up XXX phd's, XXX were determined to be from the US. Resulting in XXX Gb of (compressed) space.  

For common names (e.g.,  Williams, B), our author query returns entries from multiple authors.  To try and restrict the results to only those papers with the same author as the PhD thesis, we construct a network graph of the ADS entries using \citet{networkx}.

There can be variations in how author names are spelled, including capitalization variations (e.g., VanderPlas and Vanderplas) as well as variation in accent characters.  To ensure we match authors correctly, we reduce all author names to the format ``first initial, last name'', and consider names to match if they have a Levenshtein distance less than three.

Every ADS entry is a node in our network. We draw an edge between nodes if they meet any of the following criteria:

\begin{itemize}
\item{If the affiliations (of the possible thesis author) match, and the entries have publication years within 2 years or each other.}
\item{If the entries have 3 or more authors in common in their author lists.}
\item{If there are two authors and they are identical on the two entries.}
\item{If the information contained in the abstracts are similar enough (ratio $> 0.5$) as measured by XXX-magic fucking black boxXXX TfidfVectorizer from \citet{scikit-learn}}
\item{If the information contained in the titles of the entries are similar enough.}
  \item{If the relevant author's name matches exactly, and the entries share two keywords.}
\end{itemize}

For our affiliation matching, we consider the affiliations to match if one affiliation is contained in the other.  For example, ``University of Washington'' is considered a match to ``Department of Astronomy, University of Washington, Box 351580'' but not ``Washington University''.  We also compare the two affiliation strings with the python difflib SequenceMatcher, and consider any affiliations that have a similarity ratio greater than 0.7 as matching.  For example ``Berkeley'' and ``UC Berkeley'' have a ratio of 0.8, and would thus be considred matching.

This is an attempt at codifying the common-sense process one would take to infer if it is the same author on two papers. With author's being able to identify their papers with services such as ORCID\footnote{\url{http://orcid.org/}}, it should soon be possible to build a network of papers known to be linked to a single author as a training set for machine learning algorithms.  

As the final step, we select only the entries which are linked to the PhD thesis of interest.  Some example ADS entry network graphs are shown in Figure~\ref{fig:example_networks}.

Once we have selected only those entries we believe are connected to the thesis author, we demand that at least one of the entries be in a peer-reviewed astronomy journal or a contribution at a major astronomy conference to eliminate physics thesis entries.  The list of publications we use to label someone as an astronomer are listed in Table~\ref{table:journal_list}. We consider an ADS listed publication to be a match if any of our selected journals are contained in the publication string, e.g., ``Astrophysical Journal Letters'' would count as a match because it contains ``Astronomical Jounal'' which is in our list.  This step is necissary to remove PhD authors who would be better classified as Physists, Geophysists, Planeteray Scientists, etc. 

We do not limit our search to only first-author publications for either the publication graph construction or determining if the author meets our criterion for being labled an ``astronomer''.


The ADS publication must match at least one of:
\begin{deluxetable}{ll}
\tabletypesize{\small}% \footnotesize \scriptsize}
\tabletypesize{\small}
%\rotate
\tablewidth{0pt}
%\tablenum{num}
%\tablecolumns{num}
%\tableheadfrac{num}
\tablecaption{Publications Used to Classify Authors As Astronomers}
\tablehead{ \colhead{Journal} & \colhead{abbr.} \label{table:journal_list}}
\startdata
Astronomy and Astrophysics & A\&A \\
Astrophysical Journal & AJ \\
Astronomical Journal & ApJ \\
Monthly Notices of the Royal Astronomical Society & MNRAS 
\enddata
\end{deluxetable}


Thus, our definition of a US PhD astronomer is effectively a person who received a PhD from an institution in the United States and has published at least one paper in a major peer-reviewed English-language astronomy journal.

We record the thesis author's most recent ADS entry date (need not be a peer-reviewed paper)  and a bunch of other stuff.

We also flag if there are any other thesis entries (from any year) that match the last name, first initial of the thesis entry so that we can compare the full database to those with more unique names.

One known issue is that ADS does not contain all Astronomy PhD dissertations. We found that XXX is not listed in ADS (see github issue).


\begin{figure*}
  \plottwo{../../new_plots/single_example_network_0.pdf}{../../new_plots/single_example_network_1.pdf}\\
    \plottwo{../../new_plots/single_example_network_2.pdf}{../../new_plots/single_example_network_3.pdf}
        \caption{Examples of network graphs constructed to find papers linked to individual PhD thesis entries in ADS.   (a) Network of ADS entries with the same author as \citet{Yoachim07} (46 entries, 44 linked to the PhD), (b) Network for \citet{Bellm2011} (106 entries, 99 linked),(c) Network for \citet{Williams02}, (315 papers, 270 linked) (d) Network for \citet{Williams11} (158 papers, 113 linked).  Note, none of the linked papers for the two Williams, B PhDs overlap. \label{fig:example_networks}}
\end{figure*}

\begin{figure}
  \plotone{../../new_plots/single_example_network_4.pdf}
  \caption{Example of an ADS networks that suffers from under-linking.  (e) The under-linked network of \citet{Capelo12}.  A single ADS entry in the future could link a network back together.}
\end{figure}

% Do any of those Williams papers overlap? NO not a one! Wow. Must be using middle initials.

Possible ways we could fail to recognize an author has published for as long as they actually have (i.e., networks are under-linked):
\begin{itemize}
\item{If someone simultaneously changes institution and area of study}
  \item{If an astronomer leaves the field for an extended time and re-joins in a different field}
\item{If an astronomer changes their name (although authors can notify ADS of name changes)}
  \item{}
\end{itemize}

Ways we could mistakenly claim an author has published for longer than they actually have, we construct networks which suffer from over-linking:
\begin{itemize}
\item{If two authors have similar names and overlap at the same institution}
\item{If two authors have similar names and similar reference lists in their papers}
  \item{If two authors have similar names and similarly named co-authors}
\end{itemize}

Thus there are two types of errors that can skew our results, over-linking our networks, connecting ADS entries that are not actually by the same author and under-linking, where we fail to recognize entries that are by the same author.  To look for the effects of over- and under-linking, we flag authors where the ADS database only lists one PhD thesis as matching their last name, and first initial. These represent less-common names that should suffer much lower rates of over-linking.

Using the less-common name subset, we can also test for under-linking by assuming the names are unique and all the ADS entries returned for a name search should be linked.  The results of the unique name tests are shown in Figure~\ref{fig:unames}. We find little evidence of under-linking. Comparing the less-common name results changes the retention rate by only $\sim4$\%. When we test for under-linking, we find recent PhDs could be suffering from under-linking at the 5-8\% level, while older PhDs suffer under-linking at lower rates (2-4\%).  This should be slightly expected, as younger astronomers typically change institutions, and many will have not yet published papers that link back to their previously established cloud.  The important aspect is that the relative over- and under-linking between PhD cohorts is similar.  Thus, the recent dramatic (20\%) drop we observe in astronomer retention is real and not an artifact of how we assign ADS entries to authors.

%\begin{figure*}
%  \plotone{../../plots/span_linkCheck.pdf}
%  \caption{Solid lines show the residuals when comparing to retention curves constructed with unique names (e.g., a sub-sample that should suffer from little to no over-linking).  The dashed lines shows the comparision to unique names when the unique name group is assumed to have a completely linked network (e.g., a subsample that suffers from little to no under-linking).  \label{fig:unames}}
%  \end{figure*}


It is worth emphasizing that while there are undoubtedly some cases where we have not accurately recorded an astronomer, we do not suffer the selection biases that go with survey-based studies and that our errors should be similar across PhD cohorts, thus we can draw strong conclusions about the relative differences between PhD classes.

\section{Observations}

To test how much we are affected by over-linking ADS entries, we compare the retention curves of all astronomers to those with unique PhD names.  xxx-should also then take the unique names and see how it changes if I use the last entry rather than the last linked entry.


\begin{figure}
  \plotone{../../new_plots/single_phdsperyear.pdf}
  \caption{Number of US Astronomy PhD thesis entries found in ADS per year. \label{fig:phdperyear}}
\end{figure}


\begin{figure*}
  \plotone{../../new_plots/single_rc_eb.pdf}
  \caption{The fraction of Astronomy PhDs still active as first authors.  Error bars show ranges computed by comparing the curves to those of just authors with unique names and uique names where all records are assumed to be linked.   \label{fig:retentiont_curve}}
\end{figure*}

%\begin{figure*}
%  \plotone{../../plots/span_linkCheck_errorbars.pdf}
%  \caption{Plot of how long PhD astronomers stay active in the ADS archive.  Error bars are set by comparing to a subset of astronomers with more unique names as outlined in \S\ref{sec:db_construct}.   \label{fig:active_curves}}
%  \end{figure*}


%\begin{figure*}
%  \plotone{../../plots/span_linkCheck_errorbars_1stA.pdf}
%  \caption{Like Figure~\ref{fig:active_curves}, but now using the most recent connected paper where the author is the first author.  The dashed lines show the curves from Figure~\ref{fig:active_curves} for comparison. The downturn in the older curves (1999-2004) could be an artifact caused by astronomers who lead a paper only every few years, e.g., two years from now those curves could rise back up.  \label{fig:1stA_active}}
%\end{figure*}

%\begin{figure}
%  \plotone{../../plots/single_staff_frac.pdf}
%  \caption{The fraction to PhDs who are still apearing in ADS entries, but no longer as first authors. \label{fig:non_1st_frac}}
%\end{figure}



Figure~\ref{fig:retention_curve} is particularly disturbing, less than 50\% of the 2009 to 2012 cohorts are still apearing as first authors on ADS entries. Note, these are any entries, we have not limited it to peer-reviewed papers.  This implies that most post-2008 USA-PhD astronomers now spend more time in graduate school than they do leading their own research projects post-graduation. Imagine if a corporation had a training program that lasted 3 months, and most of the employees who completed it left after 3 months post-training.  This seems an odd place in the career path to place a bottleneck.

There is a potential bias if authors pubilsh as first authors on intervals longer than a year.  If that is the case, we would expect the last few points of each cohort curve in Figure~\ref{fig:retention_curve} to be slightly depressed.  XXX--go through and compute the typical time gap between linked 1st author papers.  

Currently, 30\% of PhDs are immediatly leaving the field, double the fraction typical pre-2008.  

%Figure~\ref{fig:non_1st_frac} shows the fraction of PhDs that are still in ADS, but not as first authors.  Each cohort shows a sharp rise in the past 2-3 years. We suspect this could be an artifact where after 8-15 years, some active astronomers have moved to mentoring roles where they do not publish as first authors every year.  Thus, if we re-ran the experiment several years from now, those sharp rises would shift to the right.

%This seems a less likely explination for the recent PhD cohorts, and a better explination could be that they have left the field, but they are still being listed as authors on projects they contributed to months to years ago.


%\begin{figure}
%  \plotone{../../plots/single_retention_by_class.pdf}
%  \caption{The 4, 6, and 10 year retention fractions as a function of PhD cohort year.\label{fig:retention}}
%\end{figure}

Figure~\ref{fig:active_curves} shows that recent PhDs are leaving the field much faster than a edcade ago.  The values of 70-80\% retention after 4-6 years may not sound so bad.  Figure~\ref{fig:1stA_active} perhaps helps explain what is going on. While recent PhDs are still apearing in the ADS archive at the 70-80\% level, they are still apearing as the first-author only 40-50\% of the time. A possible explination is that recent PhDs are leaving the field and their former collaborators are slowly finishing projects which they contributed to.

% Another possibility is that few PhDs have first-author ADS entries every year. The cohorts from the 90's and early 00's all see marked downturns at the

Note:  The curves looked weird if we included years before 1997. This could be because the job market was diffferent at that time, or it could be that ADS completeness changes. On their FAQ \footnote{\url{http://adsabs.harvard.edu/abs\_doc/faq.html}}, ADS notes they transitioned to getting abstracts directly from journals in the middle of 1995.


\section{Discussion}

{\bf{The Astronomy labor market is not suffering from a funding crisis, but rather a funding {\emph{allocation}} crisis.}} A simple thought experiment can help illustrate the problem. What would happen if the NSF was able to double it's Astronomy budget? Much of that money would be distributed as grants, that would then be used to hire graduate students and post docs. Very little of the money would go to funding permanent positions. Increasing federal astronomy funding actually exacerbates PhD over-production. 

With a large increase in funding from XXXX to XXX, driven largely by NASA, there has been an increase in Astronomy PhD production with little to no increase in long-term positions for these PhDs.  This needs to be recognized as a failure of Astronomy funding agencies, as they paid to develop a talented workforce with no plans to retain that talent. Similarly, we must hold professional organizations responsible for failing to advocate for a funding stream allocated for the long-term health of the field.  

The failure to retain early-career Astronomers is analogous to building 4 new telescopes, when one knows there will only be enough funding to operate 2.  Such a lack of planning and wasteful use of research funding would not be tolerated with hardware, and it should not be tolerated with meatware. 

There is also a mostly invisible effect on the field when the career potential drops, the best and brightest may chose to never enter the field.  The recent popping of the law school bubble %xxx-see \url{http://www.theatlantic.com/business/archive/2012/04/the-wrong-people-have-stopped-applying-to-law-school/255685/} and \url{http://www.lsac.org/lsacresources/data/lsats-administered}, \url{http://www.slate.com/blogs/moneybox/2014/08/21/law_school_applications_more_students_with_high_lsats_are_applying.html} for a starting place.


Given the sharp recent change in PhD retention seen in Figures~\ref{fig:1stA_active} and~\ref{fig:retention}, it is important to also consider the resulting impact the change will have on recruiting new graduate students. We never see the students who chose to not apply for graduate school. This is an important observational bias.  If astronomy PhD career prospects start to resemble Art History PhD career prospects, we should expect the technical skills of the two groups of graduate students to become more similar.  With poor career prospects, astronomy risks losing the ability to recruit top students. Law schools recently saw a drop in top student applications with the collapse of the legal job market (see \url{http://www.theatlantic.com/business/archive/2012/04/the-wrong-people-have-stopped-applying-to-law-school/255685/}).  

There seems to be little hope that university departments will undertake a grass-roots effort to improve the career pipeline. Thus, it seems to fall to the funding agencies to enforce better behavior.  One approach could be to require a minimum ratio of permanent-to-temporaty PhDs (and PhDs in training) in a department before that department can recieve federal funding.  This prevents reseach universities from running degree-mill type programs, or from relying too heavily on adjunts for teaching.  This also encentivises states to fund colleges at a reasonable level to maintain access to federal funding.  Another common proposed solution is to move federal funding away from univeristy departments and more towards permanent positions at national laboratories and observatories.


\subsection{Future Work}

The ability to identify individuals via their PhD and track their career progress via their apperances in ADS entries opens up a number of possible future studies including
\begin{itemize}
\item{Extending this analysis to countries beyond the US.}
\item{For many author networks, it should be possible to programatically guess the author's gender, alowing one to compare career trajectories based on gender.}
\item{One could analyze the network of PhD granting institutions and final hiring institutions to rank the prestige of graduate programs as in \citet{Clausete15}.}
\item{The technique of linking publication hsitories to PhDs can easily be extended to other fields.}
\end{itemize}

\section{Conclusions}
Using the bibliographic information in ADS, we find that retention of new American Astronomy PhDs is at a 15+ year low.  


%--------------------------------------
\acknowledgments
Some work for this paper was done at the 2016 American Astronomical Association Hack Day. Thanks to sponsors of AAS Hack Day 2016, The Large Synoptic Survey Telescope and Northrop Grumman.

This research has made use of NASA's Astrophysics Data System. Very heavy use. XXX-Add thanks to the ADS staff that bumped up my limit.

\bibliography{bib}




\end{document}

%%  LocalWords:  Metcalfe PhDs Charfman github SAO pre networkx ORCID
%%  LocalWords:  scikit Bellm sim XXXX meatware AAS
