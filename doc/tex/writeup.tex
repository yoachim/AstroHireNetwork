
%\documentclass[12pt,preprint]{aastex}
\documentclass{emulateapj}
\usepackage{url}
%\usepackage{natbib}
%\usepackage{xspace}
\def\arcsec{$^{\prime\prime}$}
\bibliographystyle{apj}
\newcommand\degree{{^\circ}}
\newcommand\surfb{$\mathrm{mag}/\square$\arcsec}
\newcommand\Gyr{\rm{~Gyr}}
\newcommand\msun{\rm{M}_\odot}
\newcommand\kms{km s$^{-1}$}
\newcommand\al{$\alpha$}
\newcommand\ha{$\rm{H}\alpha$}
\newcommand\hb{$\rm{H}\beta$}



\shorttitle{Short Title}
\shortauthors{Yoachim et al.}

\begin{document}

\title{Publication Lifetimes of American Astronomy PhDs}

\author{Peter Yoachim\altaffilmark{1}, author 2, author 3 
  }
\altaffiltext{1}{Department of Astronomy, University of Washington, Box 351580,
Seattle WA, 98195; {yoachim@uw.edu} }




\begin{abstract}
blah blah blah, abstract

\end{abstract}
\keywords{employment}



\section{Introduction}

It is often difficult to determine the state of the Astronomy job market.  Analysis of the science job market often lumps all sciences together, in which case Biologists dominate the statistics, or, if the sciences are separated, Astronomy is usually grouped with Physics, where the number of Physisists outnumbers the Astronomers.  

\citet{Seth09} build on the work of \citet{Metcalfe08} and looks at the rise of the postdoc era in Astronomy and Astrophysics.  They show that as the infaltion-adjusted US-Astronomy budget doubled over 20 years, there was a modest increase in PhD production and a huge rise in the number of temporary postdoc positions.

In this paper, we use the ADS to explore the ramifications of that rising postdoc bubble.  While the NSF regularly surveys PhD reciepents (e.g., \url{http://nsf.gov/statistics/nsf07305/}), even with high participation rates, surveys will tend to be biased against those that leave the field since they will be harder to contact.

Thus, we propose to use the publication record of PhDs to track the overall health of the field.  

Unfortunately, some astronomers still believe that ``... the [astronomy faculty job] market is no worse or better than it is has been for at least a decade or two'' (via Charfman \url{http://jjcharfman.tumblr.com/}).  The goal of this paper is to check this statement and quantify the astronomy job market by looking at the publication footprint PhD Astronomers leave in the ADS system. 

XXX-description of ADS, how it includes peer-reviewed papers as well as things like conference proceedings, posters, grants, etc.

\section{Database Construction}

All the code used for this paper is available on github\footnote{\url{https://github.com/yoachim/AstroHireNetwork}}

We use the excelent Python Module to Interact with NASA's ADS that Doesn't Suck™\footnote{\url{https://github.com/andycasey/ads}} to programatically access records from the SAO/NASA Astrophysics Data System XXX--figure out proper way to cite code. 

Our procedure for building a database of Astronomy and Astrophysics PhDs is as follows.  We select all PhD Thesis records for a given year and screen to include only those records that have an affiliation in the United States.

For each US PhD, we then query ADS for all entries with the PhD author's last name and first initial with a date limit of 7 years pre-PhD to the present (Jan 2016).

For common names (e.g., Jones, R and Williams, B), our author query returns entries from multiple authors.  To try and restrict the results to only those papers with the same author as the PhD thesis, we construct a network graph using \citet{networkx}.

Every ADS entry is a node in our network. We draw an edge between nodes if they meet any of the following criteria:
\begin{itemize}
\item{If the affiliation of the thesis author matches, and the year of the entry is within 2 years. XXX-describe affil match fuzzy string, plus if one string is contained in the other.}
\item{If the entries have 3 or more authors in common in their author lists.}
\item{If there are two authors and they are identical on the two entries.}
\item{If the information contained in the abstracts are similar enough as measured by XXX-magic fucking black boxXXX TfidfVectorizer from \citet{scikit-learn}}
  \item{If the information contained in the titles of the entries are similar enough.}
\end{itemize}

This is a rather simle codification of the common-sense process one would take to guess if it is the same author on two papers. With services such as ORCID\footnote{\url{http://orcid.org/}}, it should be possible to build a training set of papers known to be linked to a single author as a training set for machine learning algorithms.  

We then select only the entries which are linked to the PhD thesis of interest.  Some example ADS entry network graphs are shown in Figure~XXX.

Once we have selected only those entries we believe are connected to the correct thesis author, we demand that at least one of the entries be in a peer-reviewed astronomy journal (XXX-list of journals) to eliminate physics thesis entries.

Thus, our definition of a US PhD astronomer is effectively a person who recieved a PhD from an institution in the United States and has published at least one paper in an astronomy journal.

We record the thesis author's most recent ADS entry date (need not be a peer-reviewed paper)  and a bunch of other stuff.

We also flag if there are any other thesis entries (from any year) that match the last name, first initial of the thesis entry so that we can compare the full database to those with more unique names who should suffer fewer errors in the entry network construction.

One known issue is that ADS does not contain all Astronomy PhD dissertations. We found that XXX is not listed in ADS (see github issue).


Plots to make:
\begin{itemize}
\item{Do my tests for over-linking and under-linking using unique names.}
\end{itemize}


\begin{figure*}
  \plottwo{../../plots/example_network_1.pdf}{../../plots/example_network_2.pdf}\\
    \plottwo{../../plots/example_network_3.pdf}{../../plots/example_network_4.pdf}
        \caption{Examples of network graphs constructed to find papers linked to individual PhD thesis entries in ADS.   Top Left: Network of ADS entries with the same author as \citet{Yoachim07} (45 entries, 43 linked to the PhD), Top Right: Network for \citet{Bellm2011} (103 entries, 97 linked), Bottom Left: Network for \citet{Williams02}, (313 papers, 268 linked) Bottom Right: Network for \citet{Williams11} (157 papers, 111 linked).  Note, none of the linked papers for the two Williams PhDs overlap. \label{fig:example_networks}}
\end{figure*}

% Do any of those Williams papers overlap? NO not a one! Wow. Must be using middle initials.





Possible ways we could fail to recognize an author has published for as long as they actually have
\begin{itemize}
\item{If someone simultaneously changes institution and area of study}
  \item{If an astronomer leaves the field for an extended time and re-joins in a different field}
\item{If an astronomer changes their name (although authors can notify ADS of name changes)}
  \item{}
\end{itemize}

Ways we could mistakenly claim an author has published for longer than they actually have
\begin{itemize}
\item{If two authors have similar names and overlap at the same institution}
\item{If two authors have similar names and similar reference lists in their papers}
  \item{If two authors have similar names and similarly named co-authors}
\end{itemize}

The overall point is that errors can happen both ways and both should be relavively rare, giving us a more accurate measurement of astronomer career lifetimes than survey results which rarely top XXX\% response rates.

It is worth emphasizing that while there are undoubtably some cases where we have not accuratly recorded an astronomer, we do not suffer the selection biases that go with survey-based studies and that our errors should be similar accross PhD cohorts, thus we can draw strong conclusions about the relative differences between PhD classes.

\section{Observations}

To test how much we are affected by over-linking ADS entries, we compare the retention curves of all astronomers to those with unique PhD names.  xxx-should also then take the unique names and see how it changes if I use the last entry rather than the last linked entry.


\begin{figure}
  \plotone{../../plots/phdsperyear.pdf}
  \caption{Number of US Astronomy PhD thesis entries found in ADS per year. \label{fig:phdperyear}}
\end{figure}

\begin{figure}
  \plotone{../../plots/active_curves.pdf}
  \plotone{../../plots/active_curves_uname.pdf}
  \caption{ \label{fig:active_curves}}
  \end{figure}


\begin{figure}
  \plotone{../../plots/retention_by_class.pdf}
  \caption{The 4, 6, and 10 year retention fractions as a function of PhD cohort year.\label{fig:retention}}
\end{figure}


\section{Discussion}

{\bf{Astronomy is not suffering from a funding crisis, but rather a funding {\emph{allocation}} crisis.}} A simple thought experiment can help illustrate the problem. What would happen if the NSF was able to double it's Astronomy budget? Much of that money would be distributed as grants, that would then be used to hire graduate students and post docs. Very little of the money would go to funding permanent positions. Increasing federal astronomy funding would actually exaserbate PhD overproduction. 

With a large increase in funding from XXXX to XXX, driven largly by NASA, there has been an increase in Astronomy PhD production with little to no increase in long-term positions for these PhDs.  This needs to be recognized as a failure of Astronomy funding agencies, as they paid to develop a talented workforce with no plans to retain that talent. Similarly, we must hold professional organizations responsible for failing to advocate for a funding stream allocated for the long-term health of the field.  

The failure to retain early-career Astronomers is analagous to building 4 new telescopes, when one knows there will only be enough funding to opperate 2.  Such a lack of planning and wasteful use of research funding would not be tolerated with hardware, and it should not be tolerated with meatware. 



%\begin{figure}
%\epsscale{.5}
%\caption{ \emph{continued}.}
%\end{figure}


\section{Conculsions}
Using the ADS, we find that retention of new Astronomy PhDs is at a 15+ year low.  


%--------------------------------------
\acknowledgments
Thanks everyone. Thanks to sponsors of AAS Hack Day 2016.

This research has made use of NASA's Astrophysics Data System. Very heavy use.

\bibliography{bib}




\end{document}
